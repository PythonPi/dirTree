%	Brief summary of using the dirtree package to visually represent a directory structure
%	Written by Edward McCarthy

%	Declare Document Class and options
%	Dirtree will work with any document class
\documentclass[12pt,a4paper]{article}

%	Dirtree package
%	This package enables you to graphically represent directory structure
%	in your LaTeX document
\usepackage{dirtree}

%	Courier font
%	This package loads the courier font to allow the dirtree package to
%	use. Personally this looks better than the default font
\usepackage{courier}


%	Begin document
\begin{document}

%	Dirtree syntax
%	This is the simple way to add directory structure to a document.
%	To start using the dirtree comand you must start with "\dirtree{%"
%	Leaving out the "%" will cause on error on build as the first character it looks for is a "."
%	Syntax for declaring a file or directory is ".<level><space><text node>.<space>"
\dirtree{%
	.1 /.				%	The first line should always be declared as ".1 /." declaring the root
	.2 Folder1.			%	Subsequent files/folders can start at level 1 or 2, to be within root it should be at level 2
	.2 Folder2.
	.3 Folder2\_File1.	%	As the level increases, the text node for that level will be within the previous level at the higher level eg. 2 is within 1, 3 is within 2, 4 is within 3
	.3 Folder2\_File2.	%	Anything stated at the same level as the previous node will be shown at the same level as the previous node, not within it
	.2 Folder3.			%	As this is LaTeX use of underscores must be properly escaped in the text part of each node
	.3 Folder3\_Folder1.
	.4 Folder3\_Folder1\_File1.
	.3 Folder3\_File1.
}
%	To end the dirtree block simply close the brace
	
%	End of document
\end{document}